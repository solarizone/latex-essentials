% https://www.overleaf.com/learn/latex/Theorems_and_proofs

% Theorem styles

% definition boldface title, Roman body. Commonly used in definitions, conditions, problems and examples.
% plain boldface title, italicized body. Commonly used in theorems, lemmas, corollaries, propositions and conjectures.
% remark italicized title, Roman body. Commonly used in remarks, notes, annotations, claims, cases, acknowledgments and conclusions.

\documentclass{article}
\usepackage[english]{babel}
\usepackage{amsthm}

\theoremstyle{definition}
\newtheorem{definition}{Definition}[section]

\theoremstyle{remark}
\newtheorem*{remark}{Remark}

\begin{document}
\section{Introduction}
Unnumbered theorem-like environments are also possible.

\begin{remark}
This statement is true, I guess.
\end{remark}

And the next is a somewhat informal definition

\begin{definition}[Fibration]
A fibration is a mapping between two topological spaces that has the homotopy lifting property for every space \(X\).
\end{definition}
\end{document}