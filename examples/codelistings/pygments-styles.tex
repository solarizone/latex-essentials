% Pygments styles: pygmentize -L style

\begin{filecontents*}[overwrite]{sample.py}
  # This is an example
  from typing import Iterator
  
  class Math:
    @staticmethod
    def fib(n: int) -> Iterator[int]:
      """Fibonacci series up to n."""
      a, b = 0, 1
      while a < n:
        yield a
        a, b = b, a + b
  
  result = sum(Math.fib(42))
  print(f"The answer is {result}")
\end{filecontents*}

\documentclass{article}

\usepackage{filecontents}
\usepackage{xcolor}
\usepackage{minted}
\usepackage{pgffor}
\usepackage{hyperref}
\hypersetup{
    colorlinks,
    citecolor=black,
    filecolor=black,
    linkcolor=black,
    urlcolor=black
}

\title{Pygments Styles}

\begin{document}

  \maketitle

  \tableofcontents

  \foreach \style/\bgcolor in {
    default/lime!15,
    github-dark/black!85,
    nord/black!85,
    monokai/black!85,
    vim/black!85,
    rrt/black!85,
    xcode/lime!15
  } {
    \section*{\style}%不编号
    \addcontentsline{toc}{section}{\style}%加入 toc
    % inputminted 的 options 使用循环变量需要特殊处理, 见:
    % https://tex.stackexchange.com/questions/127897/how-to-use-a-macro-inside-includegraphics-options
    \edef\options{style=\style,bgcolor=\bgcolor,frame=single,xleftmargin=0em,xrightmargin=0em}
    \expandafter\inputminted\expandafter[\options]{python}{sample.py}
  }

\end{document}