\chapter{字体}

\newcommand{\sampletext}{世界自由、正义与和平的基础}

{\LaTeX} 排版过程中要注意正确配置字体,否则会出现 \mintinline{text}{'Font not found!'} 错误。
常见的错误主要是:不同平台的字体不一样,如 macOS 平台编译通过的文件在 Debian 平台就可能出现错误;
或者字体所在目录不在搜索路径上
\footnote{
关于 {\LaTeX} 中西字体排版,请参考:在 {\LaTeX} 中使用 OpenType 字体系列:
\href{https://stone-zeng.github.io/2018-08-08-use-opentype-fonts}{I},
\href{https://stone-zeng.github.io/2019-07-06-use-opentype-fonts-ii}{II},
\href{https://stone-zeng.github.io/2020-05-02-use-opentype-fonts-iii/}{III}。
}。

\section{字体的基本知识}

\subsection{介绍}
现代 TEX 引擎(包括 X⁠E⁠TEX、Lua­TEX 和 Ap­TEX)已经全面支持使用 OpenType 字体,因而可以很方便地实现类似 Microsoft Word、Adobe InDesign 等软件的效果。

OpenType 字体格式由微软和 Adobe 联合开发,它有以下特点:

\begin{itemize}
  \item 字体编码基于 Unicode
  \item 描述字体轮廓时,既可以用三次 Bézier 曲线(PostScript 轮廓),也可以用二次 Bézier 曲线(TrueType 轮廓)
  \item 支持合字(也叫连字,ligature)、小型大写字母(small caps)、多种数字样式、上下标、上下文替换等高级字体排印功能,并可用于复杂语言排版
  \item 在 Windows、macOS、Linux 等多种平台下均可使用
\end{itemize}

OpenType 字体的文件扩展名可为 .Ttf、.ttf、.ttc,它们的区别如下:

\begin{itemize}
  \item .Ttf:单个字体,使用 PostScript 轮廓
  \item .ttf:单个字体,使用 TrueType 轮廓
  \item .ttc:字体集合(TrueType/OpenType Collection),在单个文件中打包多个字体,允许使用 TrueType 或 PostScript 轮廓
\end{itemize}

LATEX 中使用 OpenType 字体,主要依靠下列宏包:

\begin{itemize}
  \item fontspec:通用字体选取
  \item CTeX:LATEX 中文排版框架,依据引擎的不同,底层分别利用 xeCJK 宏包(X⁠E⁠TEX)和 luatexja 宏包(Lua­TEX)来处理
  \item unicode-math:实验性的 Unicode 数学排版功能
\end{itemize}

考虑到 Ap­TEX 仍处于开发状态,尚未提供良好的用户接口,因此下文仅适用于 X⁠E⁠TEX 和 Lua­TEX 引擎。

\subsection{字体的分类}

大体上说,西文字体可以分为衬线体(serif font)和无衬线体(sans-serif font)两大类。

\subsubsection{衬线体}

所谓“衬线”,是指笔画末端的一种装饰细节,一般认为起源于古罗马的石刻拉丁字母。衬线据说有引导视线的作用,因而往往被用作书籍、文章等的正文字体。根据产生年代以及风格样式,又可以细分为几类。

\subsubsection{无衬线体}

顾名思义,无衬线体就是指没有衬线的字体,它在近现代才得到广泛的发展与运用。早些年,屏幕分辨率远达不到印刷质量,衬线等细节很难反映出来,因此无衬线体在屏幕、网页显示上大行其道。

\subsubsection{等宽字体}

等宽字体与比例字体相对,其中的所有字母、符号均有相同的宽度。严格来说,等宽字体并不能单独作为一个分类,但由于其用法比较特殊(现代主要用于计算机程序的排版),这里还是把它单独列出来。

\subsubsection{变体}

直立、正体的字型通常被称为罗马体(Roman type)。除此之外,一套完整的字体往往还具有若干其他字型。

\begin{itemize}
  \item 意大利体(italic type):字形稍向右倾,笔画带有手写体的风格。它不是罗马体的简单倾斜
  \item 斜体(oblique/slant type):大部分是原字形的简单倾斜,少数设计精良的字体(如 Univers、Helvetica 等)也会再额外进行视觉修正
  \item 字重(weight):表示字体的粗细,除了常规体(regular),还有粗体(bold)、细体(light),以及半粗、半细、超粗、超细、特粗、特细等。现代字体也流行以数字区分不同字重,典型例子如 Univers
  \item 小型大写(small caps):顾名思义,形状是大写字母,但尺寸略小(一般接近小写字母)。西文中,全大写字母组成的单词(如 WHO)常用小型大写的形式
\end{itemize}

\subsection{中文字体}

中文字体原则上应该与西文字体平行列出。但出于历史原因,现代可用的中文字体数量远小于西文字体,也更少有细致的分类方案。

正文中常用的中文字体主要有宋、黑、楷、仿四种。括号中给出的是对应的日文名称
\footnote{方正书宋简体不能正确显示这些日文名称,此处使用了 Noto Serif SC 字体。}。

\begin{itemize}
  \item 宋体({\notoserifsc 明朝体}):实际上诞生于明朝,因而也叫明体。特点是笔画硬朗、横细竖粗,笔画末端带有类似「衬线」的装饰三角形。宋体习惯用于正文排版,适合与衬线的西文字体搭配
  \item 黑体({\notoserifsc ゴシック体}):笔画粗细变化较小,横竖对比也较小,常与无衬线西文字体搭配
  \item 楷体({\notoserifsc 楷書体}):来源于传统书法,字形端庄,对应与西文字体中的手写体
  \item 仿宋({\notoserifsc 宋朝体}):源自于宋朝的刻书字体,但实际成形于民国初年。特点是兼有宋体的结构与楷体的笔画,较为清秀挺拔
\end{itemize}

CTeX 宏默认的字库配置基于以下逻辑:

\begin{itemize}
  \item 宋体对应到 \mintinline{text}{\rmfamily} 的 \mintinline{text}{\upshape}
  \item 黑体(或微软雅黑、苹方等现代黑体)对应到 \mintinline{text}{\sffamily} 的 \mintinline{text}{\upshape}
  \item 楷体对应到 \mintinline{text}{\rmfamily} 的 \mintinline{text}{\itshape}
  \item 仿宋对应到 \mintinline{text}{\ttfamily} 的 \mintinline{text}{\upshape}
  \item 如果以上某种字体有相应的粗体,则将其对应到 \mintinline{text}{\bfseries},如果没有则不做特殊处理;特别地,加粗宋体如果不存在的话,则会改用黑体(但不使用现代黑体)
  \item 最后使用 \mintinline{text}{\setCJKfamilyfont} 命令定义额外的的字体命令,如 \mintinline{text}{\songti}、\mintinline{text}{\heiti} 等;此时如果有隶书和圆体,也会用同样方式定义字体命令
\end{itemize}

\section{下载和安装字体}

texlive 自带的中文字体比较有限,如其自带 Fandol 楷体,该字库的“仓”误为“仑”,故不用。

可以从方正字体\footnote{其它选择: Adobe Fonts 或 Google Fonts}下载方正字体:

\url{https://www.foundertype.com/index.php/FindFont/index}

从Google Fonts 下载 Noto Sans SC 和 Noto Serif SC 字体:

\url{https://fonts.google.com/noto/fonts}

\subsection{Debian}

字体可以安装在下列目录中:

\begin{itemize}
  \item \url{/usr/share/fonts}
  \item \url{/usr/local/share/fonts}
  \item \url{~/.fonts}
\end{itemize}

\subsection{macOS}

字体可以安装在下列目录中:

\begin{itemize}
  \item \url{/System/Library/Fonts}
  \item \url{/Library/Fonts}
  \item \url{~/Library/Fonts}
\end{itemize}

\section{修改环境变量 \protect\verbum{OSFONTDIR}}

修改 \url{/usr/local/texlive/2021/texmf.cnf}(根据实际安装的 texlive 版本),添加:

\subsection{Debian}

\begin{minted}{text}
OSFONTDIR = /usr/share/fonts//;/usr/local/share/fonts//;~/.fonts//
\end{minted}

\subsection{macOS}

\begin{minted}{text}
OSFONTDIR = /System/Library/Fonts//;/Library/Fonts//;~/Library/Fonts//
\end{minted}

\subsection{设置字体}

{\LaTeX} 使用本地字库,或者在系统字体目录中安装相应的字体。
其实,{\LaTeX} 可以实现直接读取字体文件(如存储在项目的\url{fonts/}目录下)来进行文档的渲染工作,也就是项目本身含有字体文件
\footnote{注意:中文字体文件都比较大,会使得项目非常臃肿}。

\begin{minted}{tex}
  %设置英文字体
  \setmainfont[
      Path=fonts/,
      BoldFont=NotoSerif-Bold.otf,
      ItalicFont=NotoSerif-RegularItalic.otf,
      BoldItalicFont=NotoSerif-BoldItalic.otf,
  ]{NotoSerif-Regular.otf}
  \setsansfont[
      Path=fonts/,
      BoldFont=NotoSans-Bold.otf,
      ItalicFont=NotoSans-RegularItalic.otf,
      BoldItalicFont=NotoSans-BoldItalic.otf,
  ]{NotoSans-Regular.otf}
  \setmonofont[
      Path=fonts/
  ]{NotoSans-Regular.otf}
   
  %设置中文字体
  \setCJKmainfont[
      Path=fonts/,
      BoldFont=NotoSerifSC-Bold.otf,
      ItalicFont=NotoSerifSC-Regular.otf,
      BoldItalicFont=NotoSerifSC-Bold.otf
  ]{NotoSerifSC-Regular.otf}
  \setCJKsansfont[
      Path=fonts/,
      BoldFont=NotoSansSC-Bold.otf,
      ItalicFont=NotoSansSC-Regular.otf,
      BoldItalicFont=NotoSansSC-Bold.otf
  ]{NotoSansSC-Regular.otf}
  \setCJKmonofont[
      Path=fonts/
  ]{NotoSansSC-Regular.otf}
\end{minted}

\section{中文字体}

中文字体与西文字体有很大区别,中文一般没有西文对应的成套字体,如粗体、斜体等。
衬线(Serif)是字形笔画的起始段与末端的装饰细节部分,有衬线体如Times New Roman,无衬线体(Sans serif)如 Arial。
中文同样也有衬线体(如宋体)和无衬线体(如黑体)。

\subsection{罗马字体族 Roman family}

这是正文的默认字体,也可以使用 \mintinline{text}{\textrm{文字内容}} 或 \mintinline{text}{{\rmfamily 文字内容}} 重新设置文字的字体.

\begin{itemize}
  \item Normal: {\rmfamily \mdseries \sampletext}
  \item Bold: {\rmfamily \bfseries \sampletext}
  \item Italic: {\rmfamily \itshape \sampletext}
  \item BoldItalic: {\rmfamily \bfseries \itshape \sampletext}
\end{itemize}

\subsection{无衬线字体族 Sans serif family}

使用 \mintinline{text}{\textsf{文字内容}} 或 \mintinline{text}{{\sffamily 文字内容}}设置文字的字体.

\begin{itemize}
  \item Normal: {\sffamily \mdseries \sampletext}
  \item Bold: {\sffamily \bfseries \sampletext}
  \item Italic: {\sffamily \itshape \sampletext}
  \item BoldItalic: {\sffamily \bfseries \itshape \sampletext}
\end{itemize}

\subsection{打字机字体族 Typewriter family}

通常代码抄录使用该字体族,使用 \mintinline{text}{\texttt{文字内容}} 或 \mintinline{text}{{\ttfamily 文字内容}} 设置文字的字体.

\begin{itemize}
  \item Normal: {\ttfamily \mdseries \sampletext}
  \item Bold: {\ttfamily \bfseries \sampletext}
  \item Italic: {\ttfamily \itshape \sampletext}
  \item BoldItalic: {\ttfamily \bfseries \itshape \sampletext}
\end{itemize}

\subsection{中文字体族的定义与使用}

使用 \mintinline{text}{\setCJKfamily{字体名称}{字体文件名}} 定义中文字体族
\footnote{这里字体名称采用了中文,也可以使用字母,如 songti,fangsong 等},如:

\begin{minted}{tex}
  \setCJKfamilyfont{宋体}{方正书宋简体.TTF}
  \setCJKfamilyfont{圆体}{方正准圆简体.TTF}
  \setCJKfamilyfont{仿宋}{方正仿宋简体.TTF}
  \setCJKfamilyfont{楷体}{方正楷体简体.TTF}
  \setCJKfamilyfont{黑体}{方正黑体简体.TTF}
  \setCJKfamilyfont{隶书}{方正隶书简体.TTF}
  \setCJKfamilyfont{行楷}{方正行楷简体.TTF}
  \setCJKfamilyfont{行书}{字酷堂松雪行书简体.TTF}
\end{minted}

使用 \mintinline{text}{\CJKfamily{字体名称}{文字内容}} 设置文字的字体。

\begin{table}[!h]
\begin{center}
\caption{中文字体族}
\begin{tabular}{cc}
  \toprule
  字体名称 &  示例\\
  \midrule
  %宋体 & \CJKfamily{宋体}{\sampletext}\\
  圆体 & \CJKfamily{圆体}{\sampletext}\\
  仿宋 & \CJKfamily{仿宋}{\sampletext}\\
  楷体 & \CJKfamily{楷体}{\sampletext}\\
  %黑体 & \CJKfamily{黑体}{\sampletext}\\
  隶书 & \CJKfamily{隶书}{\sampletext}\\
  %行楷 & \CJKfamily{行楷}{\sampletext}\\
  %行书 & \CJKfamily{行书}{\sampletext}\\
  \bottomrule
\end{tabular}
\end{center}
\end{table}